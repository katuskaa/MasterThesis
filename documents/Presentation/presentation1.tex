% Copyright 2004 by Till Tantau <tantau@users.sourceforge.net>.
%
% In principle, this file can be redistributed and/or modified under
% the terms of the GNU Public License, version 2.
%
% However, this file is supposed to be a template to be modified
% for your own needs. For this reason, if you use this file as a
% template and not specifically distribute it as part of a another
% package/program, I grant the extra permission to freely copy and
% modify this file as you see fit and even to delete this copyright
% notice. 

\documentclass{beamer}
\usepackage[utf8]{inputenc}
\usepackage[T1]{fontenc}
\usepackage[slovak]{babel}
\usepackage[fleqn]{mathtools}
\usepackage{amsmath}
\usepackage{algorithm2e}
\usepackage{algorithmic}
\usepackage{listings}
\usepackage{color}

% There are many different themes available for Beamer. A comprehensive
% list with examples is given here:
% http://deic.uab.es/~iblanes/beamer_gallery/index_by_theme.html
% You can uncomment the themes below if you would like to use a different
% one:
%\usetheme{AnnArbor}
%\usetheme{Antibes}
%\usetheme{Bergen}
%\usetheme{Berkeley}
%\usetheme{Berlin}
%\usetheme{Boadilla} 1
%\usetheme{boxes} 2
%\usetheme{CambridgeUS}
%\usetheme{Copenhagen}
%\usetheme{Darmstadt}
%\usetheme{default}
%\usetheme{Frankfurt}
%\usetheme{Goettingen}
%\usetheme{Hannover}
%\usetheme{Ilmenau}
%\usetheme{JuanLesPins}
%\usetheme{Luebeck}
\usetheme{Madrid}
%\usetheme{Malmoe}
%\usetheme{Marburg}
%\usetheme{Montpellier}
%\usetheme{PaloAlto}
%\usetheme{Pittsburgh}
%\usetheme{Rochester}
%\usetheme{Singapore}
%\usetheme{Szeged}
%\usetheme{Warsaw}

\title{Optimization of an abductive reasoner for description logics}

% A subtitle is optional and this may be deleted
\subtitle{Master thesis}

\author{Katarína Fabianová \\ Adviser: RNDr. Martin Homola, PhD. \\ Consultant:  Mgr. Júlia Pukancová}
% - Give the names in the same order as the appear in the paper.
% - Use the \inst{?} command only if the authors have different
%   affiliation.

\date{FMFI, 2018}
% - Either use conference name or its abbreviation.
% - Not really informative to the audience, more for people (including
%   yourself) who are reading the slides online

% Delete this, if you do not want the table of contents to pop up at
% the beginning of each subsection:
%\AtBeginSubsection[]
%{
%	\begin{frame}<beamer>{The system overview}
%		\tableofcontents[currentsection,currentsubsection]
%	\end{frame}
%}

% Let's get started
\begin{document}

\begin{frame}
  \titlepage
\end{frame}

%\begin{frame}{The system overview}
%	\tableofcontents
	% You might wish to add the option [pausesections]
%\end{frame}

% Section and subsections will appear in the presentation overview
% and table of contents.

\begin{frame}{Aims}
	\begin{itemize}
		
		\item {
			Descripton logic
		}
		
		\item {
			ELK reasoner
		}
		
		\item {
			Abduction
		}
				
		\item {
			Minimal Hitting Set (Reiter)
		}
										
		\item {
			Implementation of Reiter's algorithm with optimizations
		}
		
		\item {
			Semantic minimality
		}
				
		\item {
			Evaluation of results (ELK/Conclude)
		}
										
	\end{itemize}
\end{frame}


\begin{frame}{Description logic}
  \begin{itemize}
	
	\item {
		family of knowledge representation languages
	}
	
	\item {
		every description logic has different expression
	}
	
	\item {
		each type of expression supports different constructors
	}
	
	\item {
		we will use DL $\mathcal{EL}$ and DL $\mathcal{EL{+}{+}}$
	}

  \end{itemize}
\end{frame}

\begin{frame}{Description logic (DL)}{Syntax}
	\begin{itemize}
	\item {
		important terms: individual, concept (atomic or complex), role
	}
	\item {
		DL is shaped by 3 mutually disjoint sets:
		\[ N_{I} = \{ a,b,c... \} \]
		\[ N_{C} = \{ A,B,C... \} \]
		\[N_{R} = \{ R_{1}, R_{2}, R_{3} \} \]
	}
	\item {
		DL consists of following constructors:
		\[ \neg, \sqcup, \sqcap, \forall, \exists \]
		DL conceptualization: Everybody who is sick, is not happy.
		\[ Sick \sqsubseteq \neg Happy \text{ (axiom)} \]
	}
	\end{itemize}
\end{frame}

\begin{frame}{Description logic}{Ontology}
	\begin{itemize}
				
		\item {
			ontology describes relationships between entities in a specific area
		}
		
		\item {
			contains knowledge base $\mathcal{KB} = (\mathcal{T}, \mathcal{A})$	
		}
		
		\item {
			$\mathcal{T}$ stands for TBox, $\mathcal{A}$ stands for ABox
		}
		
		\item {
			TBox contains all axioms that model ontology
		}
		
		\item {
			ABox creates database, it contains individuals
		}
		
		\[ 
		\mathcal{KB} = \left\{
		\begin{tabular}{l c}
		$Sick \sqsubseteq \neg Happy$ \\
		$mary: Sick$ \\
		\end{tabular}
		\right \}
		\]

	\end{itemize}
\end{frame}

\begin{frame}{Reasoning}
	\begin{itemize}
		\item {
			Reasoning problems: consistency, satisfiability, inference
		}
		
		\item {
			Algorithm: Tableau algorithm
		}
		
		\item {
			Aims: finding model, classification
		}
		
		\item {
			Current reasoners: Elk, Conclude, Fact++, Hermit, Pellet ...
		}
	\end{itemize}
\end{frame}

\begin{frame}{ELK reasoner}
	\begin{itemize}
		\item {
			ELK is a reasoner in DL $\mathcal{EL{+}{+}}$, earlier DL $\mathcal{EL}$
		}
		
		\item {
			ELK reasoning in polynomial time
		}
		
		\item {
			Communication through OWL API in Java
		}
			
		\item {
			ELK works with ontologies (model retrieving, classification)
		}
	\end{itemize}
\end{frame}

\begin{frame}{Abduction}
	\begin{itemize}
		
		\item {
			Knowledge base and observation is known
		}
		
		\item {
			Search for explanations
		}
		
		\[ 
		\mathcal{KB} = \left\{
		\begin{tabular}{l c}
		$Sick \sqsubseteq \neg Happy$ \\
		\end{tabular}
		\right \}
		\]
		
		\[ \mathcal{O} = \{ mary: \neg Happy \} \]
		
		\item {
			We use minimal HS algorithm to find minimal explanations
		}
		
		\item {
			Algorithm finds this explanation: 
			\[ \mathcal{E}_{1} = \{ mary: Sick \} \]
		}
	\end{itemize}
\end{frame}

\begin{frame}{Reiter's algorithm: Minimal hitting set}
	\begin{itemize}
		
		\item {
			Reiter's algorithm computes minimal hitting sets
		}
		
		\item {
			Definitions: Hitting set, HS-tree
		}
		
		\item {
			Algorithm: Generate pruned HS-tree
		}
		
	\end{itemize}
\end{frame}

\begin{frame}{Progress}
	\begin{itemize}
		\item {
			Theoretical part: 
			\begin{itemize}
				\item Descripton logic
				\item ELK reasoner
				\item Abduction							
				\item Minimal Hitting Set
			\end{itemize}	
		}
	\end{itemize}		

	\begin{itemize}
		\item {
			Practical part: 
			\begin{itemize}
				\item Project base with libraries
				\item running OWL API
			\end{itemize}
		}
	\end{itemize}
\end{frame}

\begin{frame}{What is next?}
	\begin{itemize}
		
		\item {
			Implementation of Reiter's algorithm with optimizations
		}

		\item {
			Semantic minimality
		}
				
		\item {
			Evaluation of results (ELK/Conclude)
		}
		
	\end{itemize}
\end{frame}


\begin{frame}{References}{Articles}
	\begin{itemize}
	\item {
		Yevgeny Kazakov, Markus Krötzsch, František Simančík. ELK Reasoner: Architecture and Evaluation
	}
	
	\item {
		Júlia Pukancová, Martin Homola. Tableau-Based ABox Abduction for the 		$\mathcal{ALCHO}$ Description Logic
	}
	
	\item {
		Raymond Reiter. A Theory of Diagnosis from First Principles
	}
	
	\item {
		Russell Greiner, Barbara A. Smith, Ralph W. Wilkerson. A Correction to the Algorithm in Reiter's Theory of Diagnosis
	}

	\item {
		Franz Wotawa. A variant of Reiter’s hitting-set algorithm
	}
	\end{itemize}
\end{frame}


\begin{frame}
	\[\text{\textbf{Thank you for your attention}}\]
\end{frame}

%\section{Second Main Section}

%\subsection{Another Subsection}

%\begin{frame}{Blocks}
%\begin{block}{Block Title}
%You can also highlight sections of your presentation in a block, with it's own title
%\end{block}
%\begin{theorem}
%There are separate environments for theorems, examples, definitions and proofs.
%\end{theorem}
%\begin{example}
%Here is an example of an example block.
%\end{example}
%\end{frame}

% All of the following is optional and typically not needed. 
%\appendix
%\section<presentation>*{\appendixname}
%\subsection<presentation>*{For Further Reading}

%\begin{frame}[allowframebreaks]
%  \frametitle<presentation>{For Further Reading}
    
%  \begin{thebibliography}{10}
    
%  \beamertemplatebookbibitems
  % Start with overview books.

%  \bibitem{Author1990}
%    A.~Author.
%    \newblock {\em Handbook of Everything}.
%    \newblock Some Press, 1990.
    
%  \beamertemplatearticlebibitems
  % Followed by interesting articles. Keep the list short. 

%  \bibitem{Someone2000}
%    S.~Someone.
%    \newblock On this and that.
%    \newblock {\em Journal of This and That}, 2(1):50--100,
%    2000.
%  \end{thebibliography}
%\end{frame}

\end{document}


